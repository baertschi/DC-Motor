%% Dokumenteinstellungen %%%%%%%%%%%%%%%%%%%%%%%%%%%%%%%%%%%%
\documentclass[a4paper,oneside,12pt,ngerman]{scrartcl}

%% Deutsche Anpassungen %%%%%%%%%%%%%%%%%%%%%%%%%%%%%%%%%%%%%
\usepackage[ngerman]{babel}
\usepackage[T1]{fontenc}
\usepackage[ansinew]{inputenc}
\usepackage{lmodern} %Type1-Schriftart f�r nicht-englische Texte
\usepackage{booktabs}	% sch�nere tabellen

%% Packages f�r Grafiken & Abbildungen %%%%%%%%%%%%%%%%%%%%%%
\usepackage{graphicx} %%Zum Laden von Grafiken
%\usepackage{subfig} %%Teilabbildungen in einer Abbildung
%\usepackage{tikz} %%Vektorgrafiken aus LaTeX heraus erstellen


%% Packages f�r Formeln %%%%%%%%%%%%%%%%%%%%%%%%%%%%%%%%%%%%%
\usepackage{amsmath}
\usepackage{amsthm}
\usepackage{amsfonts}


%% Andere Packages %%%%%%%%%%%%%%%%%%%%%%%%%%%%%%%%%%%%%%%%%%
%\usepackage{a4wide} %%Kleinere Seitenr�nder = mehr Text pro Zeile.
\usepackage{fancyhdr} %%Fancy Kopf- und Fu�zeilen
%\usepackage{longtable} %%F�r Tabellen, die eine Seite �berschreiten
\usepackage{lastpage}
\usepackage[raggedright]{subfigure}
\usepackage[final]{pdfpages}
\includepdfset{pages=-,noautoscale}

%%%%%%%%%%%%%%%%%%%%%%%%%%%%%%%%%%%%%%%%%%%%%%%%%%%%%%%%%%%%%
%% TODO
%%%%%%%%%%%%%%%%%%%%%%%%%%%%%%%%%%%%%%%%%%%%%%%%%%%%%%%%%%%%%
% 
% 
%%%%%%%%%%%%%%%%%%%%%%%%%%%%%%%%%%%%%%%%%%%%%%%%%%%%%%%%%%%%%



%%%%%%%%%%%%%%%%%%%%%%%%%%%%%%%%%%%%%%%%%%%%%%%%%%%%%%%%%%%%%
%% Optionen / Modifikationen
%%%%%%%%%%%%%%%%%%%%%%%%%%%%%%%%%%%%%%%%%%%%%%%%%%%%%%%%%%%%%
\input{Einstellungen}

% Formeln r�misch nummerieren
\renewcommand{\theequation}{\Roman{equation}} 

% "Formel" statt "Gleichung"
\def\equationname{Formel}

%%%%%%%%%%%%%%%%%%%%%%%%%%%%%%%%%%%%%%%%%%%%%%%%%%%%%%%%%%%%%
%% DOKUMENT
%%%%%%%%%%%%%%%%%%%%%%%%%%%%%%%%%%%%%%%%%%%%%%%%%%%%%%%%%%%%%
\begin{document}

\title{Praktikum: DC-Motor}
\date{\today}
\author{Cyril Stoller, Marcel B�rtschi}
\maketitle

%% Inhaltsverzeichnis %%%%%%%%%%%%%%%%%%%%%%%%%%%%%%%%%%%%%%%
\tableofcontents %Inhaltsverzeichnis

\vfill

\listoffigures

%\pagestyle{fancy} %%Ab hier die Kopf-/Fusszeilen: headings / fancy / ...

\newpage

\begin{abstract}
	
\begin{center}	
\textbf{Abstract}
\vspace{0.3cm}

In diesem Versuch wird ein Gleichstrommotor und ein Gleichstromgenerator ausgemessen.
\end{center}
	
\end{abstract}

\vspace{2cm}


%%%%%%%%%%%%%%%%%%%%%%%%%%%%%%%%%%%%%%%%%%%%%%%%%%%%%%%%%%%%%
%%                                                         %%
%%         Kapitel / Hauptteil des Dokumentes              %%
%%                                                         %%
%%%%%%%%%%%%%%%%%%%%%%%%%%%%%%%%%%%%%%%%%%%%%%%%%%%%%%%%%%%%%



\section{Ziel}

Dieser Bericht beinhaltet genaue Angaben zur Durchf�hrung und eine Diskussion des Versuches \emph{Mechatronik Grundlagen} im Modul \emph{???}. 

\section{Einleitung}

\subsection{Motivation}
Die in der Vorlesung erlernte Theorie soll mit diesem Praktikum in der Praxis nachvollzogen und vertieft werden. Ausserdem soll w�hrend dem Praktikum ein Laborjournal gef�hrt werden. Dies soll soweit ge�bt werden, dass es im Arbeitsalltag als Elektroingenieur zur Gewohnheit wird.

\subsection{Aufgabenstellung}
Die Aufgabenstellung ist unter \url{http://moodle.bfh.ch/} oder im Anhang zu finden.

\section{Durchf�hrung}


\subsection{Diskussion}

\section{Schlussfolgerung}


\vfill
\begin{tabular}{rr}
	\\
	\\
	\\
	\\
	\toprule
	\scriptsize{Datum und Unterschrift}	\hspace{3cm}	&	\textsc{Marcel B�rtschi}	\\
	\\
	\\
	\\
	\\
	\toprule
	\scriptsize{Datum und Unterschrift}	\hspace{3cm}	&	\textsc{Cyril Stoller}
\end{tabular}


% Der Anhang kommt auf eine neue Zeile
\newpage
% Offizielle "A Anhang" Aufz�hlungsvariante
\appendix
% Nur im Inhaltsverzeichnis hinzuf�gen (mit richtiger Seite, da vorher "\newpage"), aber kein Text
\addcontentsline{toc}{section}{Anhang}

% Quellenverzeichnis
%\addcontentsline{toc}{section}{Quellenverzeichnis}
\section{Quellenverzeichnis}
\renewcommand\refname{}

\vspace{-1cm}

\bibliographystyle{amsplain}
\bibliography{Bildquellen}

\section{Messmittelliste}
\begin{itemize}
	\item Signalgenerator: HP 33120A (MG 231-2)
	\item Speiseger�t: TTi PL320QMD (MN 221-4)
	\item Multimeter: HP 34401A (MM 319-2) (f�r RMS messungen)
	\item Phasemeter: BFH-eigenes Ger�t
	\item KO:  Agilent DSO1022A (MK220-2)
	\item Multimeter: roline RO-334 (f�r Widerst�nde und Speiseger�t Kontrolle)
\end{itemize}

\section{Messtabelle}

\begin{tabular}{|l|l|l|}
	\hline
	Frequenz in [Hz] & Amplitude EFF [mV] & Phase in [�] \\
	\hline
	\hline 200 &	87.6 & 70 \\
	\hline
	400	&84.9	&157 \\
	\hline
	600	&80.3	&307 \\
	\hline
	800	&68.3	&682 \\
	\hline
	850	&61.2	&886 \\
	\hline
	900	&49.7	&1190 \\
	\hline
	950	&29.6	&1599 \\
	\hline
	1000 &0	&1868 \\
	\hline
	1050 &-28.7	&1660 \\
	\hline
	1100 &-47	&1296 \\
	\hline
	1150 &-57.4	&1018 \\
	\hline
	1200 &-64.2	&828 \\
	\hline
	1400 &-75.5	&475 \\
	\hline
	1600 &-79.7	&338 \\
	\hline
	1800 &-81.9	&267 \\
	\hline
	2000 &-83.2 &222 \\
	\hline
	2200 &-84.1 &191 \\
	\hline
	\hline
\end{tabular}

\section{Matlab Code}

\section{Aufgabenstellung Praktikum}

\end{document}
